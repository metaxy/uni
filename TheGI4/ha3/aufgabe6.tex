\section*{Aufgabe 6}

Zu zeigen ist $(\forall n \in \mathbb{N}. P \nbisim Q) \Rightarrow P \bisim Q$\\
Annahme: $(\forall n \in \mathbb{N}. P \nbisim Q)$\\
Zu zeigen ist nun $P \bisim Q$ was aber äquivalent zu dass \\
$\mathcal{R} = \{(R,S) | R,S \in \Proc \land (\forall n \in \mathbb{N}. R \nbisim S) \}$ eine starke Bisimulation ist.\\
Unsere Annahme ist:\\
$(R,S) \in \mathcal{R}$ und $R \CCSTrans{a} R'$\\\\
Wir zeigen nun dass $\exists S' \in \Proc$ sodass $S \CCSTrans{a} S'$ und $(R',S') \in \mathcal{R}$\\
Da $\mathcal{R}$ symmetrisch ist (aufgrund dessen dass $\nbisim$ symmetrisch ist) genügt es nur diesen Fall zu zeigen.\\
Wir machen einen Widerspruchsbeweis:\\
Annahme: Es gibt kein $S'$ mit $S \CCSTrans{a} S'$ und $(R',S') \in \mathcal{R}$(1)\\
Sei $\{S_1,..,S_n\}$ die Menge der Prozesse zu denen man mit einem $a$ von $S$ kommt. Diese ist endlich, da das LTS image-finite ist.\\\\
Nun ist aber $(R,S) \in \mathcal{R}$ mit $R \CCSTrans{a} R'$, was bedeutet dass \\
$R \kbisim{n+1} S 
= (R,S) \in  \kbisim{n+1} \\ 
= (R,S) \in \mathcal{F}^{n+1}(\Proc \times \Proc)\\
= (R,S) \in \mathcal{F}(\mathcal{F}^{n}(\Proc \times \Proc))\\
= (R,S) \in \mathcal{F}(\nbisim)\\
= (R,S) \in \{(p,q) \in \Proc \times \Proc | p {\stackrel{1}{\bisim}}_{\nbisim} q \}$. \\
Daraus folgt aber dass $\forall a \in \Act . \forall p' \in Der(p,a) . \exists q' \in Der(q,a) .(p',q')  \in \nbisim$. \\
Das heißt wieder dass es ein gibt $q' \in Der(q,a) .(p',q') \in \nbisim$.\\
Dies ist aber ein Widerspruch zur Annahme(1), dass es $\exists n \in \mathbb{N}. \lnot (R' \nbisim S')$.\\
