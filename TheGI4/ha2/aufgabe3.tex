\section*{Aufgabe 3}
\subsection*{a)}

Z.z $\leq$ ist eine partielle Ordnung auf $B$.\\
Es genügt zu zeigen dass $\leq$ reflexiv, antisymetrisch und transitiv ist.\\

Reflexiv\\
Sei $f \in B$ beliebig. Dann gilt $f \leq f$ da $\Inv{f} \subseteq \Inv{f}$ (1)\\

Antisymetrisch \\
Es muss gelten $\forall f,g \in B : f \leq g \land g \leq f \rightarrow f =g$.\\
Sei $f,g \in B$ beliebig. 
Annahme: $f \leq g \land g \leq f $ \\
Z.z $f =g$\\
Aus der Annamhe folgt $\Inv{f} \subseteq \Inv{g} \land \Inv{g} \subseteq \Inv{f} $\\
 $\Rightarrow \Inv{f} = \Inv{g}$\\
Dies impliziert aber auch $\InvN{f} = \InvN{g}$ da es sich um eine boolsche Funktion handelt.\\
Daraus folgt dass $f=g$ (2)\\

Transitiv\\
Es muss gelten $\forall f,g,h \in B : f \leq g \land g \leq h \rightarrow f \leq h$.\\
Sei $f,g,h \in B$ beliebig. 
Annahme: $f \leq g \land g \leq h$\\
z.Z.: $f \leq h$\\
Aus der Annahme folgt, dass  $\Inv{f} \subseteq \Inv{g} \land \Inv{g} \subseteq \Inv{h}$\\
$\Rightarrow \Inv{f} \subseteq \Inv{h}$\\
$\Rightarrow f \leq g$ (3)\\

Mit (1) und (2) und (3) folgt, dass $\leq$ eine partielle Ordung auf $B$ ist. $\blacksquare$

\subsection*{b)}
Da wir auch TheGI3 wissen dass die Menge der boolschen Funktionen über n variablen die Mächtigkeit $2^n$ hat ist $B$ endlich.
Mit Aufgabe 2 müssen wir ledlich zeigen dass$ \bigsqcup \{f,g\}$ für $f,g \in B$ existiert.

Sei $f,g \in B$ beliebig mit $f \neq g$. 
Dann existieren sowohl $\Inv{f}$ als auch $\Inv{g}$. Auch $\subseteq$ ist für diese beiden definiert, darauf folgt dass $f \leq g$ definiert ist\\
Nun gilt:
$ \bigsqcup \{f,g\}= f$ falls $f \leq g$ sonst 
$ \bigsqcup \{f,g\}= g$.$\blacksquare$


