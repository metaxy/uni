\section*{Aufgabe 6}

\subsection*{a)}
Sei $(D,\sqsubseteq)$ ein vollständiger Verband. Sei $f$ monoton.\\
Sei $A = \{x \in D | f(x) \sqsubseteq x \}$\\
Z.z $z_{min} = \bigsqcap A$ ist der kleinste Fixpunkt.\\
Wir brauchen zu zeigen, dass\\\\
\textbf{1. $z_{min}$ ist ein Fixpunkt von $f$.}\\
Da $\sqsubseteq$ antisymmetrisch ist müssen wir zeigen dass:\\
$f(z_{min}) \rr z_{min}$ (*)  und\\
$z_{min} \rr f(z_{min}) $ (**) \\
\begin{itemize}
\item(*)\\
Nach $z_{min} = \bigsqcap A$ gilt $\forall x \in A : z_{min} \rr x$\\
Da $f$ monoton ist gilt: $\forall x \in A : f(z_{min}) \rr f(x)$\\
$\Rightarrow \forall x \in A : f(z_{min}) \rr f(x)  \rr x$\\
$\Rightarrow f(z_{min})$ ist einem untere Schranke.\\
$\Rightarrow  f(z_{min}) \rr z_{min} $ da $z_{min}$ das größte untere Schranke  ist.\\

\item(**)\\
Mit (*) gilt dass $f(z_{min}) \rr z_{min}$\\
$\Rightarrow f(f(z_{min})) \rr f(z_{min})$ da $f$ monoton \\
$\Rightarrow f(z_{min}) \in A$ nach Defintion von $A$\\
$\Rightarrow z_{min} \rr f(z_{min})$ da $z_{min}$ die untere Schranke ist\\
\end{itemize}
Aus (*) und (**) folgt mit der Antisymmetrie von $\rr$ dass $z_{min}$ ein Fixpunkt ist.\\\\
\textbf{2. $z_{min}$ ist der kleinste Fixpunkt}.\\
Widerspruchsbeweis.\\
Sei $z_{min2}$ ein Fixpunkt mit $z_{min2} \sqsubseteq z_{min}$ (1).\\
Es gilt $z_{min} = \bigsqcap \{x \in D | f(x) \sqsubseteq x \}$\\
Nun ist aber $z_{min2} \in \{x \in D | f(x) \sqsubseteq x \}$ da es ein Fixpunkt ist.\\
Das aber steht im Widerspruch zu (1), wenn $z_{min}$ das Infimum ist, kann $z_{min2}$ nicht Element der Prä-Fixpunkte sein.
 
\subsection*{b)}
z.Z Aus $(D,\sqsubseteq)$ ein endlicher vollständiger Verband und $f$ monoton folgt dass, $z_{max} = f^M(\top)$ ein $M \in \mathbb{N}$ der größte Fixpunkt von $f$ ist.\\
Sei $(D,\sqsubseteq)$ ein endlicher vollständiger Verband und $f$ monoton.
z.Z $z_{max} = f^M(\top)$ ein $M \in \N$ der größte Fixpunkt von $f$.\\\\
Wir brauchen zu zeigen, dass\\\\
\textbf{1. $z_{max}$ ist ein Fixpunkt von $f$.}\\
$z_{max} = f^M(\top) = f^{M+1}(\top)$ da $\top$ das maximale Element ist und $f$ monoton ist.\\\\
\textbf{2. $z_{max}$ ist der größte Fixpunkt}.\\
Sei $z$ ein Fixpunkt.\\
Nun gilt $z \sqsubseteq \top$. Da $f$ monoton ist
$f(z) = z  \sqsubseteq f(\top)$. Wir wenden $f$ $M-1$ mal an, und wir bekommen $z  \sqsubseteq f^M(\top) = z_{max} $\\
Daraus folgt dass $ z_{max}$ der größte Fixpunkt ist.\\\\
Mit 1. und 2. folgt dass $z_{max}$ der größte Fixpunkt ist.$\blacksquare$




