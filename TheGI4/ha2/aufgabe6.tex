\section*{Aufgabe 6}

\subsection*{a)}
Seite 108

\subsection*{(b)}
z.Z Aus $(D,\sqsubseteq)$ ein endlicher vollstänger Verband und $f$ monoton folgt dass, $z_{max} = f^M(\top)$ ein $M \in \mathbb{N}$ der größte Fixpunkt von $f$ ist.\\
Sei $(D,\sqsubseteq)$ ein endlicher vollstänger Verband und $f$ monoton.
z.Z $z_{max} = f^M(\top)$ ein $M \in \N$ der größte Fixpunkt von $f$.\\

1. $z_{max}$ ist ein Fixpunkt von $f$.\\
$z_{max} = f^M(\top) = f^{M+1}(\top)$ da $\top$ das maximale Element ist und $f$ monton ist.\\

2. $z_{max}$ ist der größte Fixpunkt.\\
Sei $z$ ein Fixpunkt.\\
Nun gilt $z \sqsubseteq \top$. Da $f$ monoton ist
$f(z) = z  \sqsubseteq f(\top)$. Wir wenden $f$ $M-1$ mal an, und wir bekommen $z  \sqsubseteq f^M(\top) = z_{max} $\\
Daraus folgt dass $ z_{max}$ der größte Fixpunkt ist.




