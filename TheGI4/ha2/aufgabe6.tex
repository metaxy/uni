\section*{Aufgabe 6}

\subsection*{a)}
\textbf{1. $z_{min}$ ist ein Fixpunkt von $f$.}\\

\textbf{2. $z_{min}$ ist der kleinste Fixpunkt}.\\
Wiederspruchsbeweis.\\
Sei $z_{min2}$ ein Fixpunkt mit $z_{min2} \sqsubseteq z_{min}$ (1).\\
Es gilt $z_{min} = \bigsqcap \{x \in D | f(x) \sqsubseteq x \}$\\
Nun ist aber $z_{min2} \in \{x \in D | f(x) \sqsubseteq x \}$ da es ein Fixpunkt ist.\\
Das aber steht im Wiederspruch zu (1), wenn $z_{min}$ das Infimum ist, kann $z_{min2}$ nicht Element der Prä-Fixpunkte sein.
 
\subsection*{(b)}
z.Z Aus $(D,\sqsubseteq)$ ein endlicher vollstänger Verband und $f$ monoton folgt dass, $z_{max} = f^M(\top)$ ein $M \in \mathbb{N}$ der größte Fixpunkt von $f$ ist.\\
Sei $(D,\sqsubseteq)$ ein endlicher vollstänger Verband und $f$ monoton.
z.Z $z_{max} = f^M(\top)$ ein $M \in \N$ der größte Fixpunkt von $f$.\\\\
Wir brauchen zu zeigen, dass\\\\
\textbf{1. $z_{max}$ ist ein Fixpunkt von $f$.}\\
$z_{max} = f^M(\top) = f^{M+1}(\top)$ da $\top$ das maximale Element ist und $f$ monton ist.\\\\
\textbf{2. $z_{max}$ ist der größte Fixpunkt}.\\
Sei $z$ ein Fixpunkt.\\
Nun gilt $z \sqsubseteq \top$. Da $f$ monoton ist
$f(z) = z  \sqsubseteq f(\top)$. Wir wenden $f$ $M-1$ mal an, und wir bekommen $z  \sqsubseteq f^M(\top) = z_{max} $\\
Daraus folgt dass $ z_{max}$ der größte Fixpunkt ist.\\\\
Mit 1. und 2. folgt dass $z_{max}$ der größte Fixpunkt ist.$\blacksquare$




