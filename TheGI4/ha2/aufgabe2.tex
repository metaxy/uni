\section*{Aufgabe 2}

z.Z.: Wenn $(X,R)$ ein Verband ist mit $X$ endlich, dann ist $(X,R)$ auch ein vollständiger Verband.\\\\
Beweis:\\
Sei $(X,R)$ ein beliebiger Verband mit $X$ endlich.\\
Daraus folgt dass für beliebige $d_1, d_2 \in X$ auch $\bigsqcap \{d_1,d_2\}$ existiert.
Beweis mittels vollständiger Induktion.\\
Sei $A_i$ eine beliebige Menge mit $A_i \subseteq X$ und $\#(A_i) = i$\\\\
\textbf{Beweis der Existenz von $\bigsqcap$}\\
Induktionsanfang: $A_2 = \{d_1,d_2\}$. Nach Voraussetzung existiert $\bigsqcap \{d_1,d_2\}$.\\
Inuktionsvorausetzung(IV): $\bigsqcap A_i$ existiert.\\
Inuktionsschritt: $\bigsqcap A_{i+1}$ \\
$\bigsqcap A_{i+1} = \bigsqcap(A_{i} \cup \{d\})$ für ein $d \in A_{i+1}$\\
Falls $\bigsqcap A_{i} \sqsubseteq d$ dann ist $\bigsqcap A_{i+1} = \bigsqcap A_{i}$ (1)\\
Falls $d \sqsubseteq \bigsqcap A_{i}$ dann ist $\bigsqcap A_{i+1} = d$ (2)\\
Aus $(1)$ und $(2)$ und (IV) folgt dass $\bigsqcap A_{i+1}$ existiert (3)\\\\
\textbf{Beweis der Existenz von $\bigsqcup$}\\
Induktionsanfang: $A_2 = \{d_1,d_2\}$. Nach Voraussetzung existiert $\bigsqcup \{d_1,d_2\}$.\\
Inuktionsvorausetzung(IV): $\bigsqcup A_i$ existiert.\\
Inuktionsschritt: $\bigsqcup A_{i+1}$ \\
$\bigsqcup A_{i+1} = \bigsqcup(A_{i} \cup \{d\})$ für ein $d \in A_{i+1}$\\
Falls $\bigsqcup A_{i} \sqsubseteq d$ dann ist $\bigsqcup A_{i+1} = d$ (4)\\
Falls $d \sqsubseteq \bigsqcup A_{i}$ dann ist $\bigsqcup A_{i+1} = \bigsqcup A_{i}$ (5)\\
Aus $(4)$ und $(5)$ udn (IV) folgt dass $\bigsqcup A_{i+1}$ existiert (6)\\\\
Aus (3) und (6) folgt dass für jede $A \subseteq X$ sowohl  $\bigsqcup A$ als auch $\bigsqcap A$ existieren, $\bigsqcup A_1$  und $\bigsqcap A_1$ trivialerweiser existieren. Daraus folgt dass $(X,R)$ nach Definition 4.3 ein vollständiger Verband ist. $\blacksquare$