\documentclass[a4paper,10pt]{article}
\usepackage[utf8]{inputenc}

\usepackage[ngerman]{babel}
\usepackage{amsmath,amsfonts,amssymb}
\usepackage[arrow, matrix, curve]{xy}
\RequirePackage{etoolbox}
\usepackage{xifthen}
\usepackage{ifthen}
\usepackage{ifmtarg}
\usepackage{semantic}
\usepackage{paralist}
\usepackage{tikz}
\usetikzlibrary{arrows}


\newcommand{\Co}[1]{\overline{\mathit{#1}}}
%opening
\title{TheGI4 HA}
\author{David Konopek(349333), Paul Walger(349968), Lukas Klammt(332263)}
\usepackage{Macros}
\usepackage{Macros}
\begin{document}

\maketitle

\section*{Aufgabe 1}

\subsection*{Aufgabe 1 a)}
Sei $f_1$ die Umbennung \\ $[betritt/betreten, zuh"ort/aktiv, guteHA/erstellen, verl"asst/verlassen]$\\ \\
und $f_2$ die Umbennung \\ $[betritt/betreten, redet/aktiv, schlechteHA/erstellen, verl"asst/verlassen]$\\
 	$SG \CCSDef \Out {betreten}{}. \In {aktiv}{}.SG1 $ \\ \\
 	$SG1 \CCSDef \Out {erstellen}{}.SG2 $ \\ \\
 	$SG2 \CCSDef \Out {verlassen}{}.SG  $ \\ \\
 	 	$TG \CCSDef SG[f_1] $ \\ \\
 	 	$HG \CCSDef SG[f_2]  $ \\ \\
 	 	$T \CCSDef \Out {betreten}{}.T1$ \\ \\
 	 	$T1 \CCSDef \Out {verlassen}{}.T$ \\ \\
 	 	$R \CCSDef \Choice{TR}{HR}$ \\ \\
 	 	$TR \CCSDef \In {betreten}{}. \In {betritt}{}.\Out {zuh"ort}{}.TR1$ \\ \\
 	 	$TR1 \CCSDef \In {verl"asst}{}. \In {verlassen}{}.R$ \\ \\
 	 	$HR \CCSDef \In {betritt}{}.\Out {redet}{}.HR1$ \\ \\
 	 	$HR1 \CCSDef \In {verl"asst}{}.R$ \\ \\
 	 	$U \CCSDef \Res *{\Par {R} \Par {T} \Par {TG}{HG}}{betritt, zuh"ort, verl"asst, redet, betreten, verlassen}  $ \\ \\
 	 	Anmerkung: \\ SG steht f"ur Studentengruppe, TG f"ur Tutoriumsgruppe, HG f"ur Hausaufgabengruppe, T f"ur Tutor, R f"ur Raum, TR f"ur Tutoriumsraum, HR f"ur Hausaufgabenraum und U f"ur Universit"at. \\
\\ \\ \\


\subsection*{Aufgabe 1 b)}
Sei $\mathcal{X} = \{betritt, zuh"ort, verl"asst, redet, betreten, verlassen\}$\\
Sei \begin{compactitem}
 \item $ P_1 =  (R | T | TG | HG) \setminus \mathcal{X}$
 \item $ P_2 =  (TR1 | T1 | SG1[f_1] | HG) \setminus \mathcal{X}$
 \item $ P_3 =  (TR1 | T1 | SG2[f_1] | HG) \setminus \mathcal{X}$
 \item $ P_4 =  (HR1 | T | TG | HG[f_2]) \setminus \mathcal{X}$
 \item $ P_5 =  (HR1 | T | TG | HG[f_2]) \setminus \mathcal{X}$
\end{compactitem}
Nun ist
\begin{align*}
LTS_U & = (Proc, Act, \{  \CCSTrans {a}  | a \in Act \})\\
\Proc &= \{ P_1,P_2,P_3,P_4,P_5 \} \\
\MC{A} &= \lbrace schlechteHA, guteHA \rbrace \\
\Act &= \lbrace \tau \rbrace \cup \MC{A} \cup \Co{\MC{A}}\\
\CCSTrans {\tau} &= \{(P_1,P_2),(P_1,P_4),(P_3,P_1),(P_5,P_1) \}
\\
\CCSTrans {schlechteHA} &= \emptyset
\\ 
\CCSTrans {\overline{schlechteHA}}  &= \{ (P_4,P_5)\}
\\ 
\CCSTrans {guteHA} &= \emptyset 
\\
\CCSTrans {\overline{guteHA}} &= \{ (P_2,P_3)\}
\end{align*}

\section*{Aufgabe 2}

z.Z.: Wenn $(X,R)$ ein Verband ist mit $X$ endlich, dann ist $(X,R)$ auch ein vollständiger Verband.\\\\
Beweis:\\
Sei $(X,R)$ ein beliebiger Verband mit $X$ endlich.\\
Daraus folgt dass für beliebige $d_1, d_2 \in X$ auch $\bigsqcap \{d_1,d_2\}$ existiert.
Beweis mittels vollständiger Induktion.\\
Sei $A_i$ eine beliebige Menge mit $A_i \subseteq X$ und $\#(A_i) = i$\\\\
\textbf{Beweis der Existenz von $\bigsqcap$}\\
Induktionsanfang: $A_2 = \{d_1,d_2\}$. Nach Voraussetzung existiert $\bigsqcap \{d_1,d_2\}$.\\
Inuktionsvorausetzung(IV): $\bigsqcap A_i$ existiert.\\
Inuktionsschritt: $\bigsqcap A_{i+1}$ \\
$\bigsqcap A_{i+1} = \bigsqcap(A_{i} \cup \{d\})$ für ein $d \in A_{i+1}$\\
Falls $\bigsqcap A_{i} \sqsubseteq d$ dann ist $\bigsqcap A_{i+1} = \bigsqcap A_{i}$ (1)\\
Falls $d \sqsubseteq \bigsqcap A_{i}$ dann ist $\bigsqcap A_{i+1} = d$ (2)\\
Aus $(1)$ und $(2)$ und (IV) folgt dass $\bigsqcap A_{i+1}$ existiert (3)\\\\
\textbf{Beweis der Existenz von $\bigsqcup$}\\
Induktionsanfang: $A_2 = \{d_1,d_2\}$. Nach Voraussetzung existiert $\bigsqcup \{d_1,d_2\}$.\\
Inuktionsvorausetzung(IV): $\bigsqcup A_i$ existiert.\\
Inuktionsschritt: $\bigsqcup A_{i+1}$ \\
$\bigsqcup A_{i+1} = \bigsqcup(A_{i} \cup \{d\})$ für ein $d \in A_{i+1}$\\
Falls $\bigsqcup A_{i} \sqsubseteq d$ dann ist $\bigsqcup A_{i+1} = d$ (4)\\
Falls $d \sqsubseteq \bigsqcup A_{i}$ dann ist $\bigsqcup A_{i+1} = \bigsqcup A_{i}$ (5)\\
Aus $(4)$ und $(5)$ udn (IV) folgt dass $\bigsqcup A_{i+1}$ existiert (6)\\\\
Aus (3) und (6) folgt dass für jede $A \subseteq X$ sowohl  $\bigsqcup A$ als auch $\bigsqcap A$ existieren, $\bigsqcup A_1$  und $\bigsqcap A_1$ trivialerweiser existieren. Daraus folgt dass $(X,R)$ nach Definition 4.3 ein vollständiger Verband ist. $\blacksquare$
\section*{Aufgabe 3}
\subsection*{a)}

Z.z $\leq$ ist eine partielle Ordnung auf $B$.\\
Es genügt zu zeigen dass $\leq$ reflexiv, antisymetrisch und transitiv ist.\\\\
\textbf{Reflexiv\\}
Sei $f \in B$ beliebig. Dann gilt $f \leq f$ da $\Inv{f} \subseteq \Inv{f}$ (1)\\\\
\textbf{Antisymetrisch \\}
Es muss gelten $\forall f,g \in B : f \leq g \land g \leq f \rightarrow f =g$.\\
Sei $f,g \in B$ beliebig. 
Annahme: $f \leq g \land g \leq f $ \\
Z.z $f =g$\\
Aus der Annamhe folgt $\Inv{f} \subseteq \Inv{g} \land \Inv{g} \subseteq \Inv{f} $\\
 $\Rightarrow \Inv{f} = \Inv{g}$\\
Dies impliziert aber auch $\InvN{f} = \InvN{g}$ da es sich um eine boolsche Funktion handelt.\\
Daraus folgt dass $f=g$ (2)\\\\
\textbf{Transitiv\\}
Es muss gelten $\forall f,g,h \in B : f \leq g \land g \leq h \rightarrow f \leq h$.\\
Sei $f,g,h \in B$ beliebig. 
Annahme: $f \leq g \land g \leq h$\\
z.Z.: $f \leq h$\\
Aus der Annahme folgt, dass  $\Inv{f} \subseteq \Inv{g} \land \Inv{g} \subseteq \Inv{h}$\\
$\Rightarrow \Inv{f} \subseteq \Inv{h}$\\
$\Rightarrow f \leq g$ (3)\\\\
Mit (1) und (2) und (3) folgt, dass $\leq$ eine partielle Ordung auf $B$ ist. $\blacksquare$

\subsection*{b)}
Da wir aus TheGI3 wissen dass die Menge der boolschen Funktionen über n variablen die Mächtigkeit $2^n$ hat ist $B$ endlich.
Mit Aufgabe 2 müssen wir lediglich zeigen dass $ \bigsqcup \{f,g\}$ für $f,g \in B$ existiert.

Sei $f,g \in B$ beliebig mit $f \neq g$. 
Dann existieren sowohl $\Inv{f}$ als auch $\Inv{g}$. Auch $\subseteq$ ist für diese beiden definiert, darauf folgt dass $f \leq g$ definiert ist\\
Nun gilt:
$ \bigsqcup \{f,g\}= f$ falls $f \leq g$ sonst 
$ \bigsqcup \{f,g\}= g$ $\blacksquare$



\section{Aufgabe 4}

\subsection{Aufgabe 4.1}

Zu zeigen ist $\Par{P}{Q} \bisim \Par{Q}{P}$ \\
Sei $\mathcal{B} = Id_{\Proc} \cup \{ (\Par{A}{B},\Par{B}{A}) | A,B \in \Proc \}$ \\

Betrachte $Id_{\Proc} \subseteq B$. Dann ist $Id_{\Proc}$ nach Definition eine Bisimulation\\

Sei $(\Par {P}{Q},\Par {Q}{P}) \in B$ \\

\textbf{Transitionen in $\Par {P}{Q}$}\\

1.Fall $COM1$ \\
\begin{displaymath}
    \inference[COM1]
    {
      P \CCSTrans{a} P'
    }
    {
        \Par{P}{Q} \CCSTrans{a} \Par{P'}{Q}
    }
\end{displaymath}

In $\Par {Q}{P}$ gibt es den Übergang.

\begin{displaymath}
    \inference[COM2]
    {
      P \CCSTrans{a} P'
    }
    {
        \Par{Q}{P} \CCSTrans{a} \Par{Q}{P'}
    }
\end{displaymath}

Nach Defintion von $\mathcal{B}$ gilt, dass $(\Par{P'}{Q},\Par{Q}{P'}) \in \mathcal{B}$.\\
\\
2.Fall $COM2$ \\
\begin{displaymath}
    \inference[COM2]
    {
      Q \CCSTrans{a} Q'
    }
    {
        \Par{P}{Q} \CCSTrans{a} \Par{P}{Q'}
    }
\end{displaymath}

In $\Par {Q}{P}$ gibt es den Übergang.

\begin{displaymath}
    \inference[COM1]
    {
      Q \CCSTrans{a} Q'
    }
    {
        \Par{Q}{P} \CCSTrans{a} \Par{Q}{P'}
    }
\end{displaymath}

Nach Definition von $\mathcal{B}$ gilt, 
dass $(\Par{P}{Q'},\Par{Q'}{P}) \in \mathcal{B}$.

3.Fall $COM3$ \\
\begin{displaymath}
    \inference[COM2]
    {
      Q \CCSTrans{a} Q' \, \, P \CCSTrans{a} P' 
    }
    {
        \Par{P}{Q} \CCSTrans{\tau} \Par{P'}{Q'}
    }
\end{displaymath}

In $\Par {Q}{P}$ gibt es den Übergang.

\begin{displaymath}
    \inference[COM1]
    {
      Q \CCSTrans{a} Q' \, \, P \CCSTrans{a} P' 
    }
    {
        \Par{Q}{P} \CCSTrans{a} \Par{Q'}{P'}
    }
\end{displaymath}

Nach Definition von $\mathcal{B}$ gilt, 
dass $(\Par{P'}{Q'},\Par{Q'}{P'}) \in \mathcal{B}$.\\\\

\textbf{Transitionen in $\Par {Q}{P}$}\\
TODO: umbennen\\

1.Fall $COM1$ \\
\begin{displaymath}
    \inference[COM1]
    {
      P \CCSTrans{a} P'
    }
    {
        \Par{P}{Q} \CCSTrans{a} \Par{P'}{Q}
    }
\end{displaymath}

In $\Par {Q}{P}$ gibt es den Übergang.

\begin{displaymath}
    \inference[COM2]
    {
      P \CCSTrans{a} P'
    }
    {
        \Par{Q}{P} \CCSTrans{a} \Par{Q}{P'}
    }
\end{displaymath}

Nach Defintion von $\mathcal{B}$ gilt, dass $(\Par{P'}{Q},\Par{Q}{P'}) \in \mathcal{B}$.\\
\\
2.Fall $COM2$ \\
\begin{displaymath}
    \inference[COM2]
    {
      Q \CCSTrans{a} Q'
    }
    {
        \Par{P}{Q} \CCSTrans{a} \Par{P}{Q'}
    }
\end{displaymath}

In $\Par {Q}{P}$ gibt es den Übergang.

\begin{displaymath}
    \inference[COM1]
    {
      Q \CCSTrans{a} Q'
    }
    {
        \Par{Q}{P} \CCSTrans{a} \Par{Q}{P'}
    }
\end{displaymath}

Nach Definition von $\mathcal{B}$ gilt, 
dass $(\Par{P}{Q'},\Par{Q'}{P}) \in \mathcal{B}$.

3.Fall $COM3$ \\
\begin{displaymath}
    \inference[COM2]
    {
      Q \CCSTrans{a} Q' \, \, P \CCSTrans{a} P' 
    }
    {
        \Par{P}{Q} \CCSTrans{\tau} \Par{P'}{Q'}
    }
\end{displaymath}

In $\Par {Q}{P}$ gibt es den Übergang.

\begin{displaymath}
    \inference[COM1]
    {
      Q \CCSTrans{a} Q' \, \, P \CCSTrans{a} P' 
    }
    {
        \Par{Q}{P} \CCSTrans{a} \Par{Q'}{P'}
    }
\end{displaymath}

Nach Definition von $\mathcal{B}$ gilt, 
dass $(\Par{P'}{Q'},\Par{Q'}{P'}) \in \mathcal{B}$.
\subsection{Aufgabe 4.2}
$\Res *{\Par{a.\nil}{\Co{a}}.P}{a} \bisim \tau.P$ gilt nicht.\\
Gegenbeispiel: $P \CCSDef a.\nil$\\
Unter der der Bedingung, dass weder die Aktion $a$, noch ihre co-Aktion $\Co{a}$ in $P$ enthalten ist, gilt diese Bisimulation.
%Wiederspruchsbeweis:\\
%Wir nehmen an $\Res *{\Par{a.\nil}{\Co{a}}.P}{a} \bisim \tau.P$ gilt, d.h es $\mathcal{B}$ ist eine Bisimulation mit $(\Res *{\Par{a.\nil}{\Co{a}}.P}{a},\tau.P) \in \mathcal{B}$.


\part*{TheGI4 Hausaufgabenblatt 3}
  \section*{Aufgabe 5}

\subsection*{Aufgabe 5 a)}
$
R = \{(q_1,p_3),(q_2,p_4),(q_3,p_5),(q_4,p_1),(q_7,p_2),(q_6,p_{11}),(q_5,p_{10}),$\\$(q_{12},p_6),(q_{13},p_7),(q_{11},p_9),(q_{10},p_8),(q_{9},p_9),(q_{8},p_8)\}
$\\
R ist eine starke Simulation und $(q_1, p_3) \in R$. Also wird $q_1$ von $p_3$ stark simuliert.\\\\
\subsection*{Aufgabe 5 b)}
$S = \{(p_3,q_1),(p_4,q_2),(p_5,q_3),(p_1,q_4),(p_2,q_7),(p_{11},q_6),(p_{10},q_5),$\\$(p_6,q_{12}),(p_7,q_{13}),(p_9,q_{11}),(p_8,q_{10}),(p_8,q_8),(p_9,q_9)\}$\\
S ist eine starke Simulation und $(p_3, q_1) \in S$. Also wird $p_3$ von $q_1$ stark simuliert.\\\\
\subsection*{Aufgabe 5 c)}
Aus a) folgt, dass $q_1$ von $p_3$ stark simuliert wird und aus b) folgt, dass $p_3$ von $q_1$ stark simuliert wird. Also simulieren sich $q_1$ und $p_3$ gegenseitig stark.\\\\
\subsection*{Aufgabe 5 d)}
R aus a) ist eine starke Bisimulation und $(q_1, p_3) \in R$. Also sind $q_1$ und $p_3$ stark bisimilar.\\\\
\subsection*{Aufgabe 5 e)}
$q_{12}$ und $ p_6$ sind stark bisimilar.
Sei $B = \Set{(p_6,q_{12}),(p_7,q_{13}),(p_9,q_{11}),(p_8,q_{10}),(p_9,q_9),(p_8,q_8)}$
Es bleibt zu zeigen, dass B eine starke Bisimulation ist.\\

Betrachte $(p_6, q_{12}) \in B$\\
Transitionen in $p_6$:\\
-Wenn $p_6 \CCSTrans {a} p_7$, dann $q_{12} \CCSTrans {a} q_{13}$ und $(p_7, q_{13}) \in B$.\\
-Wenn $p_6 \CCSTrans {c} p_9$, dann $q_{12} \CCSTrans {c} q_{11}$ und $(p_9, q_{11}) \in B$.\\
Transitionen in $q_{12}$:\\
-Wenn $q_{12} \CCSTrans {a} q_{13}$, dann $p_{6} \CCSTrans {a} p_{7}$ und $(p_7, q_{13}) \in B$.\\
-Wenn $q_{12} \CCSTrans {c} q_{11}$, dann $p_{6} \CCSTrans {c} p_{9}$ und $(p_9, q_{11}) \in B$.\\

Betrachte $(p_7, q_{13}) \in B$\\
Transitionen in $p_7$:\\
-Wenn $p_7 \CCSTrans {c} p_6$, dann $q_{13} \CCSTrans {c} q_{12}$ und $(p_6, q_{12}) \in B$.\\
Transitionen in $q_{13}$:\\
-Wenn $q_{13} \CCSTrans {c} q_{12}$, dann $p_{7} \CCSTrans {c} p_{6}$ und $(p_6, q_{12}) \in B$.\\

Betrachte $(p_9, q_{11}) \in B$\\
Transitionen in $p_9$:\\
-Wenn $p_9 \CCSTrans {b} p_8$, dann $q_{11} \CCSTrans {b} q_{10}$ und $(p_8, q_{10}) \in B$.\\
Transitionen in $q_{13}$:\\
-Wenn $q_{11} \CCSTrans {b} q_{10}$, dann $p_{9} \CCSTrans {b} p_{8}$ und $(p_8, q_{10}) \in B$.\\

Betrachte $(p_8, q_{10}) \in B$\\
Transitionen in $p_8$:\\
-Wenn $p_8 \CCSTrans {b} p_9$, dann $q_{10} \CCSTrans {b} q_{11}$ und $(p_9, q_{9}) \in B$.\\
Transitionen in $q_{10}$:\\
-Wenn $q_{10} \CCSTrans {b} q_{9}$, dann $p_{8} \CCSTrans {b} p_{9}$ und $(p_9, q_{9}) \in B$.\\

Betrachte $(p_9, q_{9}) \in B$\\
Transitionen in $p_9$:\\
-Wenn $p_9 \CCSTrans {b} p_8$, dann $q_{9} \CCSTrans {b} q_{8}$ und $(p_8, q_{8}) \in B$.\\
Transitionen in $q_{9}$:\\
-Wenn $q_{9} \CCSTrans {b} q_{8}$, dann $p_{9} \CCSTrans {b} p_{8}$ und $(p_8, q_{8}) \in B$.\\

Betrachte $(p_8, q_{8}) \in B$\\
Transitionen in $p_8$:\\
-Wenn $p_8 \CCSTrans {b} p_9$, dann $q_{8} \CCSTrans {b} q_{11}$ und $(p_9, q_{11}) \in B$.\\
Transitionen in $q_{18}$:\\
-Wenn $q_{8} \CCSTrans {b} q_{11}$, dann $p_{8} \CCSTrans {b} p_{9}$ und $(p_9, q_{11}) \in B$.\\

Da B eine Starke Bisimulation ist und $(p_6, q_{12}) \in B$, sind $p_6$ und $ q_{12}$ stark bisimilar.


\section*{Aufgabe 6}

Zu zeigen ist $(\forall n \in \mathbb{N}. P \nbisim Q) \Rightarrow P \bisim Q$\\
Annahme: $(\forall n \in \mathbb{N}. P \nbisim Q)$\\
Zu zeigen ist nun $P \bisim Q$ was aber äquivalent zu dass \\
$\mathcal{R} = \{(R,S) | R,S \in \Proc \land (\forall n \in \mathbb{N}. R \nbisim S) \}$ eine starke Bisimulation ist.\\
Unsere Annahme ist:\\
$(R,S) \in \mathcal{R}$ und $R \CCSTrans{a} R'$\\\\
Wir zeigen nun dass $\exists S' \in \Proc$ sodass $S \CCSTrans{a} S'$ und $(R',S') \in \mathcal{R}$\\
Da $\mathcal{R}$ symmetrisch ist (aufgrund dessen dass $\nbisim$ symmetrisch ist) genügt es nur diesen Fall zu zeigen.\\
Wir machen einen Widerspruchsbeweis:\\
Annahme: Es gibt kein $S'$ mit $S \CCSTrans{a} S'$ und $(R',S') \in \mathcal{R}$\\
Sei $\{S_1,..,S_n\}$ die Menge der Prozesse zu denen man mit einem $a$ von $S$ kommt. Diese ist endlich, da das LTS image-finite ist.\\
Nach Annahme gibt $\exists n \in \mathbb{N}. (R',S') \notin \nbisim$\\

\section*{Aufgabe 7}
Seien $A,B \subseteq 2^{\Proc}$ beliebig\\
Zu zeigen ist $\eqS{A}{B} \Rightarrow \eqO{V}{A}{B}$ für alle $V \in {\mathcal{M}}_{\{X\}}$\\
Beweis durch strukturelle Induktion:\\
Induktionsanfang:\\
1. $V := X$\\
\begin{align*}
\eqS{A}{B} & \Rightarrow \eqO{X}{A}{B} \\
\Leftrightarrow \eqS{A}{B} & \Rightarrow \eqS{A}{B}
\end{align*}
2. $V := \true$\\
\begin{align*}
\eqS{A}{B} & \Rightarrow \eqO{\true}{A}{B} \\
\Leftrightarrow \eqS{A}{B} & \Rightarrow \eqS{\Proc}{\Proc}
\end{align*}
3. $V := \false$\\
\begin{align*}
\eqS{A}{B} & \Rightarrow \eqO{\true}{A}{B} \\
\Leftrightarrow \eqS{A}{B} & \Rightarrow \eqS{\emptyset}{\emptyset}
\end{align*}\\
Induktionsvorrausetzung:\\
$\eqS{A}{B} \Rightarrow \eqO{V}{A}{B}$\\\\
Induktionsschritt:\\
4. $V := F \lor G$\\
\begin{align*}
\eqS{A}{B} & \Rightarrow \eqO{F \lor G}{A}{B} \\
\Leftrightarrow \eqS{A}{B} & \Rightarrow \eqS{\OSem{F}{A} \cup \OSem{G}{A}}{\OSem{F}{B} \cup \OSem{G}{B}}
\end{align*}
gilt da $\eqO{F}{A}{B}$ und $\eqO{G}{A}{B}$\\\\
5. $V := \poss{a}F$\\
\begin{align*}
\eqS{A}{B} & \Rightarrow \eqO{\poss{a}F}{A}{B} \\
\Leftrightarrow \eqS{A}{B} & \Rightarrow \eqS{\possDenot{a}\OSem{F}{A}}{\possDenot{a}\OSem{F}{B}}
\end{align*}
gilt da $\eqO{F}{A}{B}$ und $\possDenot{a}$ TODO


\section*{Aufgabe 8}
  \subsection*{a)}
  $X \HMmax \HMand {\HMor *{\HMand *{\necess {a} \false}{\poss {b} \true}}{\HMand *{\poss {a} \true}{\necess {b} \false}}}{\necess {\Act}X} $
  \subsection*{b)}
  $F_2$ = $\HMand {\poss {c} \true}{X}$\\
  $X \HMmin \necess {a} \HMor *{\HMor *{\necess {a}\false}{\poss{b}\true}}{\poss {\Act}X}$




\end{document}
