\section*{Aufgabe 4}

\subsection*{Aufgabe 4 a)}

Zu zeigen ist $\Par{P}{Q} \bisim \Par{Q}{P}$ \\
Sei $\mathcal{B} = Id_{\Proc} \cup \{ (\Par{A}{B},\Par{B}{A}) | A,B \in \Proc \}$ \\

Betrachte $Id_{\Proc} \subseteq B$. Dann ist $Id_{\Proc}$ nach Definition eine Bisimulation\\

Sei $(\Par {P}{Q},\Par {Q}{P}) \in B$ \\

\textbf{Transitionen in $\Par {P}{Q}$}\\

1.Fall $COM1$ \\
\begin{displaymath}
    \inference[COM1]
    {
      P \CCSTrans{a} P'
    }
    {
        \Par{P}{Q} \CCSTrans{a} \Par{P'}{Q}
    }
\end{displaymath}

In $\Par {Q}{P}$ gibt es den Übergang.

\begin{displaymath}
    \inference[COM2]
    {
      P \CCSTrans{a} P'
    }
    {
        \Par{Q}{P} \CCSTrans{a} \Par{Q}{P'}
    }
\end{displaymath}
und $(\Par{P'}{Q},\Par{Q}{P'}) \in \mathcal{B}$.\\
\\
2.Fall $COM2$ \\
\begin{displaymath}
    \inference[COM2]
    {
      Q \CCSTrans{a} Q'
    }
    {
        \Par{P}{Q} \CCSTrans{a} \Par{P}{Q'}
    }
\end{displaymath}

In $\Par {Q}{P}$ gibt es den Übergang.

\begin{displaymath}
    \inference[COM1]
    {
      Q \CCSTrans{a} Q'
    }
    {
        \Par{Q}{P} \CCSTrans{a} \Par{Q}{P'}
    }
\end{displaymath}
und $(\Par{P}{Q'},\Par{Q'}{P}) \in \mathcal{B}$.\\ \\
3.Fall $COM3$ \\
\begin{displaymath}
    \inference[COM2]
    {
      Q \CCSTrans{a} Q' \, \, P \CCSTrans{a} P' 
    }
    {
        \Par{P}{Q} \CCSTrans{\tau} \Par{P'}{Q'}
    }
\end{displaymath}

In $\Par {Q}{P}$ gibt es den Übergang.

\begin{displaymath}
    \inference[COM1]
    {
      Q \CCSTrans{a} Q' \, \, P \CCSTrans{a} P' 
    }
    {
        \Par{Q}{P} \CCSTrans{a} \Par{Q'}{P'}
    }
\end{displaymath}
und $(\Par{P'}{Q'},\Par{Q'}{P'}) \in \mathcal{B}$.\\\\

\textbf{Transitionen in $\Par {Q}{P}$}\\
1.Fall $COM1$ \\
\begin{displaymath}
    \inference[COM1]
    {
      Q \CCSTrans{a} Q'
    }
    {
        \Par{Q}{P} \CCSTrans{a} \Par{Q'}{P}
    }
\end{displaymath}

In $\Par {P}{Q}$ gibt es den Übergang.

\begin{displaymath}
    \inference[COM2]
    {
      Q \CCSTrans{a} Q'
    }
    {
        \Par{P}{Q} \CCSTrans{a} \Par{P}{Q'}
    }
\end{displaymath}
und $(\Par{Q'}{P},\Par{P}{Q'}) \in \mathcal{B}$.\\
\\
2.Fall $COM2$ \\
\begin{displaymath}
    \inference[COM2]
    {
      P \CCSTrans{a} P'
    }
    {
        \Par{Q}{P} \CCSTrans{a} \Par{Q}{P'}
    }
\end{displaymath}

In $\Par {P}{Q}$ gibt es den Übergang.

\begin{displaymath}
    \inference[COM1]
    {
      P \CCSTrans{a} P'
    }
    {
        \Par{P}{Q} \CCSTrans{a} \Par{P'}{Q}
    }
\end{displaymath}

Nach Definition von $\mathcal{B}$ gilt, 
dass $(\Par{Q}{P'},\Par{P'}{Q}) \in \mathcal{B}$.\\
3.Fall $COM3$ \\
\begin{displaymath}
    \inference[COM3]
    {
      Q \CCSTrans{a} Q' \, \, P \CCSTrans{a} P' 
    }
    {
        \Par{Q}{P} \CCSTrans{a} \Par{Q'}{P'}
    }
\end{displaymath}

In $\Par {P}{Q}$ gibt es den Übergang.
\begin{displaymath}
    \inference[COM3]
    {
      Q \CCSTrans{a} Q' \, \, P \CCSTrans{a} P' 
    }
    {
        \Par{P}{Q} \CCSTrans{\tau} \Par{P'}{Q'}
    }
\end{displaymath}

und $(\Par{P'}{Q'},\Par{Q'}{P'}) \in \mathcal{B}$.\\
Daraus folgt dass $\mathcal{B}$ eine Bisimulation ist und somit $P|Q \bisim Q|P$.
\subsection*{Aufgabe 4 b)}
$\Res *{\Par{a.\nil}{\Co{a}}.P}{a} \bisim \tau.P$ gilt nicht.\\
Angenommen $\Res *{\Par{a.\nil}{\Co{a}}.P}{a} \bisim \tau.P$\\ Dann gibt es eine starke Bisimulation $ \mathcal{R}$
Dann muss $(\Res *{\Par{a.\nil}{\Co{a}}.P}{a}, \tau.P) \in \mathcal{R}$ \\
Dann aber auch $(\Res *{\nil | P}{a}, P) \in \mathcal{R}$.\\
Sei nun $P \CCSDef a.\nil$\\
Nun aber ist $(\Res *{\Par{\nil}{a.\nil}}{a}, a.\nil) \not\in \mathcal{R}$ da da $a$ von der sichtbaren Ausführung ausgeschlossen ist. \\
Unter der der Bedingung, dass weder die Aktion $a$, noch ihre co-Aktion $\Co{a}$ in $P$ enthalten ist, gilt diese Bisimulation.


