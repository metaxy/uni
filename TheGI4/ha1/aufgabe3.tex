\documentclass[a4paper,10pt]{article}
\usepackage[utf8]{inputenc}
\usepackage{Macros}
\usepackage[ngerman]{babel}
\usepackage{amsmath,amsfonts,amssymb}

\RequirePackage{etoolbox}
\usepackage{xifthen}
\usepackage[arrow, matrix, curve]{xy}
\usepackage{semantic}
\newcommand{\Co}[1]{\overline{\mathit{#1}}}
%opening
\title{TheGI4 HA}
\author{}

\begin{document}

\maketitle


\section{Aufgabe 3}

\subsection{Aufgabe 3.1}

Zu zeigen ist $\Par{P}{Q} \bisim \Par{Q}{P}$ \\
Sei $\mathcal{B} = Id_{\Proc} \cup \{ (\Par{A}{B},\Par{B}{A}) | A,B \in \Proc \}$ \\

Betrachte $Id_{\Proc} \subseteq B$. Dann ist $Id_{\Proc}$ nach Definition eine Bisimulation\\

Sei $(\Par {P}{Q},\Par {Q}{P}) \in B$ \\

\textbf{Transitionen in $\Par {P}{Q}$}\\

1.Fall $COM1$ \\
\begin{displaymath}
    \inference[COM1]
    {
      P \CCSTrans{a} P'
    }
    {
        \Par{P}{Q} \CCSTrans{a} \Par{P'}{Q}
    }
\end{displaymath}

In $\Par {Q}{P}$ gibt es den Übergang.

\begin{displaymath}
    \inference[COM2]
    {
      P \CCSTrans{a} P'
    }
    {
        \Par{Q}{P} \CCSTrans{a} \Par{Q}{P'}
    }
\end{displaymath}

Nach Defintion von $\mathcal{B}$ gilt, dass $(\Par{P'}{Q},\Par{Q}{P'}) \in \mathcal{B}$.\\
\\
2.Fall $COM2$ \\
\begin{displaymath}
    \inference[COM2]
    {
      Q \CCSTrans{a} Q'
    }
    {
        \Par{P}{Q} \CCSTrans{a} \Par{P}{Q'}
    }
\end{displaymath}

In $\Par {Q}{P}$ gibt es den Übergang.

\begin{displaymath}
    \inference[COM1]
    {
      Q \CCSTrans{a} Q'
    }
    {
        \Par{Q}{P} \CCSTrans{a} \Par{Q}{P'}
    }
\end{displaymath}

Nach Definition von $\mathcal{B}$ gilt, 
dass $(\Par{P}{Q'},\Par{Q'}{P}) \in \mathcal{B}$.

3.Fall $COM3$ \\
\begin{displaymath}
    \inference[COM2]
    {
      Q \CCSTrans{a} Q' \, \, P \CCSTrans{a} P' 
    }
    {
        \Par{P}{Q} \CCSTrans{\tau} \Par{P'}{Q'}
    }
\end{displaymath}

In $\Par {Q}{P}$ gibt es den Übergang.

\begin{displaymath}
    \inference[COM1]
    {
      Q \CCSTrans{a} Q' \, \, P \CCSTrans{a} P' 
    }
    {
        \Par{Q}{P} \CCSTrans{a} \Par{Q'}{P'}
    }
\end{displaymath}

Nach Definition von $\mathcal{B}$ gilt, 
dass $(\Par{P'}{Q'},\Par{Q'}{P'}) \in \mathcal{B}$.\\\\

\textbf{Transitionen in $\Par {Q}{P}$}\\
TODO: umbennen\\

1.Fall $COM1$ \\
\begin{displaymath}
    \inference[COM1]
    {
      P \CCSTrans{a} P'
    }
    {
        \Par{P}{Q} \CCSTrans{a} \Par{P'}{Q}
    }
\end{displaymath}

In $\Par {Q}{P}$ gibt es den Übergang.

\begin{displaymath}
    \inference[COM2]
    {
      P \CCSTrans{a} P'
    }
    {
        \Par{Q}{P} \CCSTrans{a} \Par{Q}{P'}
    }
\end{displaymath}

Nach Defintion von $\mathcal{B}$ gilt, dass $(\Par{P'}{Q},\Par{Q}{P'}) \in \mathcal{B}$.\\
\\
2.Fall $COM2$ \\
\begin{displaymath}
    \inference[COM2]
    {
      Q \CCSTrans{a} Q'
    }
    {
        \Par{P}{Q} \CCSTrans{a} \Par{P}{Q'}
    }
\end{displaymath}

In $\Par {Q}{P}$ gibt es den Übergang.

\begin{displaymath}
    \inference[COM1]
    {
      Q \CCSTrans{a} Q'
    }
    {
        \Par{Q}{P} \CCSTrans{a} \Par{Q}{P'}
    }
\end{displaymath}

Nach Definition von $\mathcal{B}$ gilt, 
dass $(\Par{P}{Q'},\Par{Q'}{P}) \in \mathcal{B}$.

3.Fall $COM3$ \\
\begin{displaymath}
    \inference[COM2]
    {
      Q \CCSTrans{a} Q' \, \, P \CCSTrans{a} P' 
    }
    {
        \Par{P}{Q} \CCSTrans{\tau} \Par{P'}{Q'}
    }
\end{displaymath}

In $\Par {Q}{P}$ gibt es den Übergang.

\begin{displaymath}
    \inference[COM1]
    {
      Q \CCSTrans{a} Q' \, \, P \CCSTrans{a} P' 
    }
    {
        \Par{Q}{P} \CCSTrans{a} \Par{Q'}{P'}
    }
\end{displaymath}

Nach Definition von $\mathcal{B}$ gilt, 
dass $(\Par{P'}{Q'},\Par{Q'}{P'}) \in \mathcal{B}$.
\subsection{Aufgabe 3.2}
$\Res *{\Par{a.\nil}{\Co{a}}.P}{a} \bisim \tau.P$ gilt nicht.\\
Gegenbeispiel: $P \CCSDef a.\nil$\\
Unter der der Bedingung, dass weder die Aktion $a$, noch ihre co-Aktion $\Co{a}$ in $P$ enthalten ist, gilt diese Bisimulation.
%Wiederspruchsbeweis:\\
%Wir nehmen an $\Res *{\Par{a.\nil}{\Co{a}}.P}{a} \bisim \tau.P$ gilt, d.h es $\mathcal{B}$ ist eine Bisimulation mit $(\Res *{\Par{a.\nil}{\Co{a}}.P}{a},\tau.P) \in \mathcal{B}$.


\end{document}
