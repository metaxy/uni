\documentclass[a4paper,10pt]{article}
\usepackage[utf8]{inputenc}
\usepackage{Macros}
\usepackage[ngerman]{babel}
\usepackage{amsmath,amsfonts,amssymb}

\RequirePackage{etoolbox}
\usepackage{xifthen}
\usepackage[arrow, matrix, curve]{xy}
\usepackage{semantic}
%opening
\title{TheGI4 HA}
\author{}

\begin{document}

\maketitle

\begin{abstract}

\end{abstract}

\section{Aufgabe 3}

\subsection{Aufgabe 3.1}

Zu zeigen ist $ \Par {P}{Q} \bisim \Par {Q}{P}$ \\
Sei $B = Id_{\Proc} \cup \Set { (\Par {P}{Q},\Par {Q}{P})}$ \\

Betrachte $Id_{\Proc} \subseteq B$. Dann ist $Id_{\Proc}$ nach Definition eine Bisimulation\\

Betrachte $(\Par {P}{Q},\Par {Q}{P}) \in B$ \\
Transitionen in $\Par {P}{Q}$\\

1.Fall $COM_1$ \\
\begin{displaymath}
    \inference[COM1]
    {
        \inference[ACT]
        {
        }
        {
            P \CCSTrans{a} P'
        }
    }
    {
        \Par{P}{Q} \CCSTrans{a} \Par{P'}{Q}
    }
\end{displaymath}

In $\Par {Q}{P}$ gibt es den Übergang

\begin{displaymath}
    \inference[COM2]
    {
         \inference[ACT]
        {
        }
        {
            P \CCSTrans{a} P'
        }
    }
    {
        \Par{Q}{P} \CCSTrans{a} \Par{Q}{P'}
    }
\end{displaymath}

Und $(\Par{P'}{Q},\Par{Q}{P'}) \in B $

2.Fall $COM_2$ \\
\begin{displaymath}
    \inference[COM2]
    {
        \inference[ACT]
        {
        }
        {
            Q \CCSTrans{a} Q'
        }
    }
    {
        \Par{P}{Q} \CCSTrans{a} \Par{P}{Q'}
    }
\end{displaymath}

In $\Par {Q}{P}$ gibt es den Übergang

\begin{displaymath}
    \inference[COM1]
    {
         \inference[ACT]
        {
        }
        {
            Q \CCSTrans{a} Q'
        }
    }
    {
        \Par{Q}{P} \CCSTrans{a} \Par{P}{Q'}
    }
\end{displaymath}

%Und $(\Par{P'}{Q},\Par{P'}{Q}) \in Id_{\Proc}$
\end{document}
