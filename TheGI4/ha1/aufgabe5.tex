\part*{TheGI4 Hausaufgabenblatt 3}
  \section*{Aufgabe 5}

\subsection*{Aufgabe 5 a)}
$
R = \{(q_1,p_3),(q_2,p_4),(q_3,p_5),(q_4,p_1),(q_7,p_2),(q_6,p_{11}),(q_5,p_{10}),$\\$(q_{12},p_6),(q_{13},p_7),(q_{11},p_9),(q_{10},p_8),(q_{9},p_9),(q_{8},p_8)\}
$\\
R ist eine starke Simulation und $(q_1, p_3) \in R$. Also wird $q_1$ von $p_3$ stark simuliert.\\\\
\subsection*{Aufgabe 5 b)}
$S = \{(p_3,q_1),(p_4,q_2),(p_5,q_3),(p_1,q_4),(p_2,q_7),(p_{11},q_6),(p_{10},q_5),$\\$(p_6,q_{12}),(p_7,q_{13}),(p_9,q_{11}),(p_8,q_{10}),(p_8,q_8),(p_9,q_9)\}$\\
S ist eine starke Simulation und $(p_3, q_1) \in S$. Also wird $p_3$ von $q_1$ stark simuliert.\\\\
\subsection*{Aufgabe 5 c)}
Aus a) folgt, dass $q_1$ von $p_3$ stark simuliert wird und aus b) folgt, dass $p_3$ von $q_1$ stark simuliert wird. Also simulieren sich $q_1$ und $p_3$ gegenseitig stark.\\\\
\subsection*{Aufgabe 5 d)}
R aus a) ist eine starke Bisimulation und $(q_1, p_3) \in R$. Also sind $q_1$ und $p_3$ stark bisimilar.\\\\
\subsection*{Aufgabe 5 e)}
$q_{12}$ und $ p_6$ sind stark bisimilar.
Sei $B = \Set{(p_6,q_{12}),(p_7,q_{13}),(p_9,q_{11}),(p_8,q_{10}),(p_9,q_9),(p_8,q_8)}$
Es bleibt zu zeigen, dass B eine starke Bisimulation ist.\\

Betrachte $(p_6, q_{12}) \in B$\\
Transitionen in $p_6$:\\
-Wenn $p_6 \CCSTrans {a} p_7$, dann $q_{12} \CCSTrans {a} q_{13}$ und $(p_7, q_{13}) \in B$.\\
-Wenn $p_6 \CCSTrans {c} p_9$, dann $q_{12} \CCSTrans {c} q_{11}$ und $(p_9, q_{11}) \in B$.\\
Transitionen in $q_{12}$:\\
-Wenn $q_{12} \CCSTrans {a} q_{13}$, dann $p_{6} \CCSTrans {a} p_{7}$ und $(p_7, q_{13}) \in B$.\\
-Wenn $q_{12} \CCSTrans {c} q_{11}$, dann $p_{6} \CCSTrans {c} p_{9}$ und $(p_9, q_{11}) \in B$.\\

Betrachte $(p_7, q_{13}) \in B$\\
Transitionen in $p_7$:\\
-Wenn $p_7 \CCSTrans {c} p_6$, dann $q_{13} \CCSTrans {c} q_{12}$ und $(p_6, q_{12}) \in B$.\\
Transitionen in $q_{13}$:\\
-Wenn $q_{13} \CCSTrans {c} q_{12}$, dann $p_{7} \CCSTrans {c} p_{6}$ und $(p_6, q_{12}) \in B$.\\

Betrachte $(p_9, q_{11}) \in B$\\
Transitionen in $p_9$:\\
-Wenn $p_9 \CCSTrans {b} p_8$, dann $q_{11} \CCSTrans {b} q_{10}$ und $(p_8, q_{10}) \in B$.\\
Transitionen in $q_{13}$:\\
-Wenn $q_{11} \CCSTrans {b} q_{10}$, dann $p_{9} \CCSTrans {b} p_{8}$ und $(p_8, q_{10}) \in B$.\\

Betrachte $(p_8, q_{10}) \in B$\\
Transitionen in $p_8$:\\
-Wenn $p_8 \CCSTrans {b} p_9$, dann $q_{10} \CCSTrans {b} q_{11}$ und $(p_9, q_{9}) \in B$.\\
Transitionen in $q_{10}$:\\
-Wenn $q_{10} \CCSTrans {b} q_{9}$, dann $p_{8} \CCSTrans {b} p_{9}$ und $(p_9, q_{9}) \in B$.\\

Betrachte $(p_9, q_{9}) \in B$\\
Transitionen in $p_9$:\\
-Wenn $p_9 \CCSTrans {b} p_8$, dann $q_{9} \CCSTrans {b} q_{8}$ und $(p_8, q_{8}) \in B$.\\
Transitionen in $q_{9}$:\\
-Wenn $q_{9} \CCSTrans {b} q_{8}$, dann $p_{9} \CCSTrans {b} p_{8}$ und $(p_8, q_{8}) \in B$.\\

Betrachte $(p_8, q_{8}) \in B$\\
Transitionen in $p_8$:\\
-Wenn $p_8 \CCSTrans {b} p_9$, dann $q_{8} \CCSTrans {b} q_{11}$ und $(p_9, q_{11}) \in B$.\\
Transitionen in $q_{18}$:\\
-Wenn $q_{8} \CCSTrans {b} q_{11}$, dann $p_{8} \CCSTrans {b} p_{9}$ und $(p_9, q_{11}) \in B$.\\

Da B eine Starke Bisimulation ist und $(p_6, q_{12}) \in B$, sind $p_6$ und $ q_{12}$ stark bisimilar.

